\documentclass[]{article}

\usepackage[legalpaper, margin=1in]{geometry}

%opening
\title{Seattle Replication Game}
\author{}

\begin{document}

\maketitle

%\begin{abstract}

%\end{abstract}

\section{Notes on Registration and Pre-Analysis Plan}

\begin{itemize}
	\item Trial registered in the AER RCT Registry (AEARCTR-0001836) December 07, 2016. This is before the intervention start date.
	\begin{itemize}
		\item However, this did NOT include a formal pre-analysis plan yet.
		\item It does include an abstract, a description of the three treatments, a broad description of outcomes to be studied, and the experimental design. All of these match the final paper.
		\item There are two differences between the registry and the final paper: 
		\begin{itemize}
			\item One difference is ``Randomisation will take place weekly in blocks of approximately 400 women determined by the timing of requesting a new loan.'' In the final paper, the blocks were of approximately 250 women.
			\item One other difference is the planned sample size. The original sample size was 3,600, but the final paper had a sample of 3,000.
		\end{itemize}
		\item This registration did not include a power analysis (i.e., did not answer ``Minimum detectable effect size for main outcomes.'')
	\end{itemize}
	\item The formal pre-analysis plan was posted on December 11, 2017. This was before the endline surveys were completed.
	\begin{itemize}
		\item Baseline surveys were conducted between January and June 2017, so before the pre-analysis plan.
		\item Focus groups in September 2017.
		\item Endline survey October 2017 until January 2018.
	\end{itemize}
	\item Pre-analysis plan
	\begin{itemize}
		\item Reflects sample size of 3,000.
		\item Things from the baseline survey not emphasized in main analysis: business turnover and employment, goals.
		\item Hypotheses - Primary outcomes
		\begin{enumerate}
			\item Hypothesis 1: Mobile money accounts positively affect business performance (Self reported monthly business profits)
			\item Hypothesis 2: Mobile money accounts facilitate saving (Total savings (self-reported))
			\item Hypothesis 3: Mobile money accounts facilitate business investment (Value of assets used in the business)
			\item The hypothesis tests listed in Section 4.2 are as in the paper.
		\end{enumerate}
	\item Hypotheses - Secondary outcomes -- ``These outcomes are excluded from
	multiple hypothesis testing and should be considered exploratory, giving insight into additional
	effects and mechanisms.''
	\begin{enumerate}
		\item Hypothesis 4: Treatment with the mobile money accounts affects labour investment in the respondent’s business (Total hours worked in business (respondent, family members and
		workers)) - in Appendix Table A.9.
		\item Hypothesis 5: Treatment with the mobile money account affects remittances (Total remittances sent) - Appendix Table A.29.
		\item Hypothesis 6: Treatment with the mobile money account affects female empowerment (female empowerment index) - Appendix Table A.27.
		\item Hypothesis 7: Treatment with the mobile money account affects female well-being (Well-being index) - Appendix Table A.26.
		\item Hypothesis 8: Mobile money accounts affect the entrepreneur’s household income (Total household income)
		\item Hypothesis 9: Mobile money accounts affects the entrepreneur’s household wealth (Household total wealth (assets, savings, land value))
		\item Hypothesis 10: Mobile money accounts affect the entrepreneur’s household consumption (Total household consumption)
		\item Not in pre-analysis plan but in the paper: Appendix Table A.28, record-keeping outcomes; A.30, number of women in group you'd interact with; A.31, loan repayment; 
	\end{enumerate}
	\item Specifications 
	\begin{enumerate}
		\item Equation 1 is as in the final paper. Not in the main results: ``To estimate the local average treatment effect, the above equation will be estimated where assignment to treatment is replaced with actual take-up, which is instrument by assignment, giving the two-stage least squares estimator''
	\end{enumerate}
	\item Indeces
	\begin{itemize}
		\item  ``First, I will re-code all contributing outcomes so that higher values correspond to treatment effects in the same direction (“better” outcomes). Second, I will standardize the individual outcomes using the baseline mean and standard deviation for that outcome. Third, I will calculate the average of the standardized constituent outcomes, weighted by the inverse covariance matrix. Where an outcome value is missing for a respondent, I will omit this outcome from the index construction.
	\end{itemize}
	\item Heterogeneity
	\begin{itemize}
		\item Heterogeneity by all randomization variables. (Four categories.)
		\begin{itemize}
			\item In actual paper: Heterogeneity by index of self-control, rather than hyperbolic time preference dummy; similarly, index of family pressure rather than dummy for WTP to hide.
			\item But Table A.32 uses components (high profits, hide money, current loan (cf first time), hyperbolic), as well as impatient, high risk taking, high saving,
		\end{itemize}
		\item Heterogeneity by above median savings dummy, above median business assets dummy, married, above median empowerment dummy, sent a remittance dummy, family request money dummy, main savings goal is business dummy, other HH member has a business dummy.
		\begin{itemize}
			\item In paper, Table A.22 has presence of spouse (not married or spouse away from home)
			\item Table A. 32 includes high inventory,  high asset, married, high empower, sent family, family takes, saves for business, spouse or other HH member has a business, and agent nearby. 
			\item So, all of the reported heterogenetiy dummies, and more.
		\end{itemize}
		\item Says q-values will be reported for heterogeneity, adjusted based on there being 3 primary outcomes tested.
	\end{itemize}
	\end{itemize}
\end{itemize}

\section{Notes on Stratification}
\begin{itemize}
	\item From Pre-Analysis Plans
	\begin{itemize}
		\item December 11, 2017 - first pre-analysis plan
		\begin{itemize}
			\item ``All data collection is in the form of surveys collected as face-to-face interviewers. Data collection
			began with the baseline survey upon a woman applying for a loan (before randomization and loan
			disbursement) and occurs again with an endline survey upon the completion of the loan. Baseline
			surveys took place between January 2017 and June 2017. The baseline survey covered demographic
			and business characteristics, business outcomes including profit, turnover and employment, consumption, saving behaviour and goals and transfers to family and friends. There is a one-week
			delay between a woman applying for a loan and receiving it, during which time BRAC carry out
			their own loan appraisal process. The baseline was carried out alongside this process. All women
			meeting the condition that they had a mobile phone and were applying for a loan (either for the
			first time or as a repeat borrower) completed the baseline survey. This was continued until the
			sample size of 3000 borrowers was met.''
			\item ``Randomisation took place weekly in blocks of 150-200 women determined by the timing of
			requesting a new loan. Women were individually randomised into the treatment or control groups.
			This continued for approximately 5 months until the sample size of 3,000 was achieved. The randomisation was stratified by present bias and willingness-to-pay-to-hide-money, first time borrower
			with BRAC, microfinance branch and also by business profits at baseline (since Fafchamps et al.
			2011 showed heterogeneous effects of giving loans to women based on their profitability).''
		\end{itemize}
		\item December 7, 2016 - first registration 
		\begin{itemize}
			\item ``Randomisation will take place weekly in blocks of approximately 400 women determined by the timing of requesting a new loan. Loans will be disbursed weekly according to the method determined in the randomisation.''
		\end{itemize}
		\item From the published paper
		\begin{itemize}
			\item ``Randomization took place weekly in batches of 250 women determined by the timing of requesting a new loan. Within each batch, all women accepted for a loan were individually randomized into the treatment or control groups. Randomization continued for approximately 5 months until the sample size of 3,000 was achieved. Lists of treatment assignments from the randomization were sent to the BRAC branches weekly, and only women who had completed the baseline survey and had been assigned a treatment could have a loan disbursed to them.'' (pg. 1422)
			\item ``Since assignment to treatment took place over a number of months and women applied for loans on different schedules, within the same group, women would be assigned to different treatment arms at different time points or not be in the study at all'' (pg. 1422) 
			\item I'm not exactly sure what was meant by groups (can every woman in the group separately apply for a loan? what is the function of the groups?), though Footnote 10 is meant to explain some: ``On average, groups had 20 members and 5 were in the study. Due to the relatively large number of microfinance groups in the study (476), there are 48 groups that only had 1 study participant.''
			\item ``The randomization was stratified by five variables at baseline: a dummy variable capturing present bias from a multiple price list incentivized game (Harrison et al. 2002), a dummy variable capturing if the woman switched above the median in a willingness to pay to hide money from the spouse game (Almås et al. 2018) (see Section IIIA for more details on the incentivized games), a dummy variable captur- ing if the client is a first-time borrower with BRAC, the microfinance branch, and a dummy variable for above-median business profit.'' (pg 1422)
		\end{itemize}
	\end{itemize}
\end{itemize}

\end{document}
